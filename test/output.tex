\documentclass[11pt,pdflatex,makeidx]{scrbook}   % Book class in 11 points
\usepackage[margin=0.5in]{geometry}
\usepackage{color}
\usepackage{makeidx}
\usepackage{hyperref}
\usepackage{listings}
\usepackage{hyperref}
\usepackage{courier}
\lstloadlanguages{Lisp}
\lstset{frame=none,language=Lisp,
  basicstyle=\ttfamily\small,
  keywordstyle=\color{black}\bfseries,
  stringstyle=\ttfamily,
  showstringspaces=false,breaklines}
\lstnewenvironment{code}{}{}
\parindent0pt  \parskip10pt             % make block paragraphs
\raggedright                            % do not right justify
% Note that book class by default is formatted to be printed back-to-back.
\makeindex
\begin{document}                        % End of preamble, start of text.
\title{\bf Erudite output test}


\author{Mariano Montone}

\date{\today}                           %   Use current date.
\frontmatter                            % only in book class (roman page #s)
\maketitle                              % Print title page.
\tableofcontents                        % Print table of contents
\mainmatter                             % only in book class (arabic page #s)
\long\def\ignore#1{}

\section{Introduction}

This is a test of Erudite output rendering, commands and syntax elements

\section{Chunks}

This is a good chunk
\begin{code}
<<<chunk1>>>
\end{code}

This is a good chunk
\begin{code}
<<<chunk2>>>
\end{code}
This is the chunk:
\begin{code}
<<chunk2>>=
(+ 1 1)

\end{code}

\begin{code}
<<chunk4>>=
(print "Start")

\end{code}
The end
\begin{code}
<<<chunk4>>>
\end{code}

This is the factorial function:
\begin{code}
(defun factorial (n)
  (if (<= n 1)
<<<base-case>>>
<<<recursive-case>>>
      ))

\end{code}
The base case is simple, just check for \verb|n=1| less:
\begin{code}
<<base-case>>=
      1

\end{code}
The recursive step is \verb|n x n - 1|:
\begin{code}
<<recursive-case>>=
      (* n (factorial (1- n)))

\end{code}

\section{Extracts}

Extract test
This has been extracted
\begin{code}
(+ 1 2)
\end{code}

Extract 3

Start
End

\section{Includes}

Include test
This is includeA
\begin{code}
(print "include A")
(print "Include")
\end{code}
This is includeB
\begin{code}
(print "include B")
\end{code}

\section{Ignore}


\section{Conditional output}
When test
This should appear

This is latex text

\section{Erudite syntax}
\subsection{Subsection}
\subsection{Subsubsection}

\subsection{Verbatim}
\begin{verbatim}
This is in verbatim
\end{verbatim}

\subsection{Code}
\begin{code}
(defun hello-world ()
(print "Hello world"))
\end{code}

\subsection{List}
\begin{itemize}
\item First item
\item Second item
\end{itemize}

\subsection{Emphasis}
\emph{This is emphasized}

\textbf{This is in bold}

\textit{This is in italics}

\subsection{Inline verbatim}
This is in \verb|inline verbatim|

\subsection{Link}
\href{https://github.com/mmontone/erudite}{Erudite}

\subsection{Reference}
\hyperref[hello-world]{hello-world}

\subsection{Label and index}
\label{label-test}
\index{label-test}
This section is labelled
                             % ignore macro
\chapter{Index}
\printindex
\end{document}
